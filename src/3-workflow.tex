\documentclass[8pt]{beamer}
% Add references to notes
\makeatletter\defbeameroption{show only notes}[]{\beamer@notestrue\beamer@notesnormalsfalse}

% Originally from https://kbroman.org/blog/2013/10/07/better-looking-latexbeamer-slides/
\usetheme{default}
\hypersetup{pdfpagemode=UseNone} % hides bookmarks on initial view
\beamertemplatenavigationsymbolsempty % removes navigation buttons clashing with the defined slide numbers

\setbeamertemplate{sections/subsections in toc}[sections numbered] % replaces bullets in toc with numbers
\setbeamertemplate{bibliography item}{\insertbiblabel} % Numbered bibliography
\setbeamertemplate{itemize subitem}{{\textendash}} % changes bullets to \textendash
% Make bullets/nums smaller
\setbeamerfont{itemize/enumerate subbody}{size=\footnotesize}
\setbeamerfont{itemize/enumerate subitem}{size=\footnotesize}
% Slide number
 \setbeamertemplate{footline}{%
   \raisebox{5pt}{\makebox[\paperwidth]{\hfill\makebox[20pt]{\color{gray}
         \scriptsize\insertframenumber}}}\hspace*{5pt}}

\definecolor{foreground}{RGB}{255,255,255}
\definecolor{background}{RGB}{24,24,24}
\definecolor{title}{RGB}{107,174,214}
\definecolor{gray}{RGB}{155,155,155}
\definecolor{subtitle}{RGB}{102,255,204}
\definecolor{hilight}{RGB}{102,255,204}
\definecolor{vhilight}{RGB}{255,111,207}

\setbeamercolor{titlelike}{fg=title}
\setbeamercolor{subtitle}{fg=subtitle}
\setbeamercolor{institute}{fg=gray}
\setbeamercolor{normal text}{fg=foreground,bg=background}
\setbeamercolor{item}{fg=foreground} % color of bullets
\setbeamercolor{subitem}{fg=gray}
\setbeamercolor{itemize/enumerate subbody}{fg=gray}
\setbeamercolor{section in toc}{fg=foreground}
\setbeamercolor{subsection in toc}{fg=gray}


\usepackage{amsmath} % for more symbols and mafs
\usepackage{hyperref} % for links
\usepackage{fancyvrb}
\usepackage{hyperref}

\usepackage{array}
\newcolumntype{M}{>{\centering\arraybackslash}m{2.2cm}}
\newcolumntype{R}{>{\centering\arraybackslash}m{1.5cm}}

\usepackage{caption}
\captionsetup{justification=raggedleft} % right aligns multiline captions

\graphicspath{{../images/}}

\usepackage[backend=biber, sorting=none]{biblatex}
\addbibresource{../bib.bib}

\usepackage{xepersian} % must be the last package
\settextfont{XB Roya}
\setlatintextfont{Vazir}
\setdigitfont{XB Roya}
\setmonofont{Iosevka}
\usefonttheme{serif} % (Required for Persian)

\makeatletter
% Originally from http://qa.parsilatex.com/14100
% -----
% BEGIN List fix
% -----
\expandafter\let\csname beamer@@tmpop@itemize item@default\endcsname\relax
\expandafter\let\csname beamer@@tmpop@itemize subitem@default\endcsname\relax
\expandafter\let\csname beamer@@tmpop@itemize subsubitem@default\endcsname\relax

\defbeamertemplate*{itemize item}{default}{\scriptsize\raise1.25pt\hbox{\donotcoloroutermaths$\blacktriangleleft$}}
\defbeamertemplate*{itemize subitem}{default}{\tiny\raise1.5pt\hbox{\donotcoloroutermaths$\blacktriangleleft$}}
\defbeamertemplate*{itemize subsubitem}{default}{\tiny\raise1.5pt\hbox{\donotcoloroutermaths$\blacktriangleleft$}}

\patchcmd{\@listi}{\leftmargin}{\rightmargin}{}{}
\let\@listI\@listi
\patchcmd{\@listii}{\leftmargin}{\rightmargin}{}{}
\patchcmd{\@listiii}{\leftmargin}{\rightmargin}{}{}
\patchcmd{\beamer@enum@}{\raggedright}{\raggedleft}{}{}
\patchcmd{\@@description}{\raggedright}{\raggedleft}{}{}
\patchcmd{\@@description}{\leftmargin}{\rightmargin}{}{}

\renewcommand{\itemize}[1][]{
  \beamer@ifempty{#1}{}{\def\beamer@defaultospec{#1}}
  \ifnum \@itemdepth >2\relax\@toodeep\else
    \advance\@itemdepth\@ne
    \beamer@computepref\@itemdepth% sets \beameritemnestingprefix
    \usebeamerfont{itemize/enumerate \beameritemnestingprefix body}%
    \usebeamercolor[fg]{itemize/enumerate \beameritemnestingprefix body}%
    \usebeamertemplate{itemize/enumerate \beameritemnestingprefix body begin}%
    \list{\usebeamertemplate{itemize \beameritemnestingprefix item}}{\def\makelabel##1{{
      \hss\llap{{
        \usebeamerfont*{itemize \beameritemnestingprefix item}
        \usebeamercolor[fg]{itemize \beameritemnestingprefix item}##1}}
      }}
    }
  \fi
  \beamer@cramped
  \raggedleft
  \beamer@firstlineitemizeunskip
}
% -----
% END List fix
% -----
% BEGIN TOC fix
% -----
\expandafter\let\csname beamer@@tmpop@subsection in toc@default\endcsname\relax
\expandafter\let\csname beamer@@tmpop@subsubsection in toc@default\endcsname\relax
\defbeamertemplate*{subsection in toc}{default}
{\leavevmode\rightskip=1.5em\inserttocsubsection\par}

\defbeamertemplate*{subsubsection in toc}{default}
{\leavevmode\normalsize\usebeamerfont{subsection in toc}\rightskip=3em
  \usebeamerfont{subsubsection in toc}\inserttocsubsubsection\par}
% -----
% END TOC fix
% -----
\makeatother

\raggedleft % right aligns for Persian texts


\author{محمدیاسین داوده}
\title{لینوکس}
\subtitle{هرآنچه لازم است بدانید}
\date{\today}

\newcommand{\start}[1][نقشه راه]{
  \frame{\maketitle}
  \begin{frame}{#1}\tableofcontents\end{frame}
}

\newcommand{\bib}{
  \begin{LTR}
    \printbibliography[heading=none]
  \end{LTR}
}

\newcommand{\refs}{
  \section{مراجع}
  \begin{frame}[allowframebreaks]{مراجع}
    \bib
  \end{frame}
  \note{\bib}
}

\newcommand{\subt}[1]{{\footnotesize\color{subtitle}{#1}}}

\newcommand{\divider}[1]{\frame{\Huge{#1}}}

\newcommand{\subdivider}[1]{\frame{\color{hilight}\huge{#1}}}

\newcommand{\alongside}{\and\\\small\smallskip}

\newcommand{\singleton}[2][]{
  \begin{frame}{#1}
    \centering
    #2
  \end{frame}
}

\newcommand{\includetwins}[3][\textwidth]{
  \includegraphics[width=.49#1]{#2}
  \includegraphics[width=.49#1]{#3}
}
% -----
% BEGIN Code (minted)
% -----
\newenvironment{code*}[2][]{
  \VerbatimEnvironment
  \begin{LTR}
    \begin{minted}[#1, linenos, mathescape]{#2}%
    }{
    \end{minted}
  \end{LTR}
}

\newcommand{\codecaption}[1][]{\captionsetup{type=listings}\captionof{listing}{#1}}

\AtBeginEnvironment{minted}{\renewcommand{\fcolorbox}[4][]{#4}} % Disable syntax error red boxes
% -----
% END Code (minted)
% -----


\subtitle{قسمت سوم: روند کاری}
\begin{document}
\start

% فاسری
\section{محیط گرافیکی}
\divider{محیط گرافیکی}
\begin{frame}{اجزای کلی مرتبط با محیط گرافیکی}
  \begin{itemize}
    \item گرافیک سرور (\lr{X} یا \lr{Wayland}) (رابط بین درایور گرافیک و مانیتور)
    \item محیط دسکتاپ (\lr{Desktop Environment})، شامل:
    \begin{itemize}
      \item مدیر پنجره (\lr{Window Manager})
      \item کامپوزیتور (Compositor)
      \item سایر ابزارها مانند نرم‌افزارهای کاربردی، پلاگین‌های UI، نوتیفیکشن سرور، مدیر فونت
    \end{itemize}
  \end{itemize}
\end{frame}
\note{
  محیط‌های دسکتاپ صرفاً گروهی از بسته‌ها و تنظیمات آنها هستند. شما می‌توانید هر کدام را که خواستید نصب نکنید یا ویرایش کنید.

  مدیر پنجره رفتار پنجره‌ها را مدیریت می‌کند. بی‌مدیر پنجره سرور گرافیک خیلی محدود مدیریت می‌کند.
  مدیر‌های پنجره دو نوع هستند: Stacked و Tiling.
  در نوع Stacked به طور پیشفرض پنجره‌ها معلق هستند. بسیار پیش می‌آید که پنجره‌ها روی هم قرار بگیرند.
  در نوع Tiling به طور پیشفرض پنجره‌ها روی هم قرار نمی‌گیرند و غالباً بیشترین فضای ممکن را اشغال می‌کنند.

  کامپوزیتور افکت‌های گرافیکی پنجره‌ها را مدیریت می‌کند. مثلاً شفافیت، انیمیشن‌های انتقال و\ldots.
}
\subdivider{تجربه کاری شخصی‌سازی شده}
\singleton{\textit{در لینوکس سؤال «ممکن است یا نه» نیست؛ سؤال درست «بلدم یا نه» است.}}
\begin{frame}{رابط‌های گرافیکی بی‌شمار و بی‌نظیر}
  بسته به رابط گرافیکی ممکن است تجربه‌ای متفاوت از لینوکس داشته باشید.

  هر رابط کاربری با بی‌نهایت ترکیب همراه است.
  \vfill
  \textbf{رایج‌ترین محیط‌های دسکتاپ:}
  \begin{itemize}
    \item KDE / Plasma (پیشنهادی برای کاربران ویندوز)
    \item Gnome و فورک‌های آن (\lr{Cinnamon}، \lr{MATE}، \lr{Unity} و\ldots)
    \item Xfce
    \item LXDE / LXQt
    \item Pantheon
    \item Deepin DE
  \end{itemize}
\end{frame}
\note{به جز این لیست تعدادی هم WM هست که با هیچ DE به طور رسمی همراه نیستند مانند i3.}
\frame{\pic[1]{plasma_vanilla}{رابط کاربری \lr{Plasma (KDE)} با تم پیشفرض Breeze~\cite{fig:kde:plasma_vanilla}}}
\note{
  رابط پلاسمای KDE به دلیل بلوغ، پختگی و بی‌نیاز بودن به تنظیم پیشنهاد می‌شود.

  این رابط به محض نصب (\lr{Out-of-the-box}) قابل استفاده است و رابطی مشابه مایکروسافت ویندوز دارد.

  این رابط (مانند بسیاری از دیگر دسکتاپ‌های لینوکس) فوق‌العاده قابل شخصی‌سازی است. با این تفاوت که شخصی‌سازی آن بسیار آسان و ویزاردی است و احتیاج به دانش خاصی ندارد.

  شخصی‌سازی رابط‌های کاربری را از جوانب بصری اصطلاحاً رایس کردن (Ricing) می‌گویند. RICE مخفف Race Inspired Cosmetic Enhancement (تزئینات الهام گرفته از مسابقات) می‌باشد.
  تزئیناتی که قابلیتی اضافه نمی‌کند اما ظاهر ماشین را شبیه ماشین‌های مسابقه‌ای می‌کند که ظاهری خاص دارند که به خنک شدن و ایرودینامیک شدن آنها کمک می‌کند.
  پیش از این، این اصطلاح برای شخصی‌سازی و سریع‌تر نشان دادن ماشین‌های وارداتی آسیایی ارزان قیمت استفاده می‌شد.~\cite{rice}

  % فارسی
  WM پیش‌فرض پلاسما KWin (کوین /\lr{\texttt{kɛvin}}/)است.
}
\note{
  به تفاوت بین بارها، فریم پنجره‌ها، شفافیت و افکت‌ها، آیکون‌ها و رنگ‌ها دقت شود.
  آیکون‌ها، کرسر (نشانگر) موس و رنگ‌ها معمولاً در قالب پکیج‌های جدا و به صورت ماژول ارائه می‌شوند و افکت‌ها معمولاً به واسطهٔ کامپوزیتورها ساخته می‌شوند.
}
\frame{\pic[1]{plasma_glassy}{رابط کاربری شخصی‌سازی شده \lr{Plasma (KDE)} با تم Glassy روی توزیع \lr{KDE Neon}~\cite{fig:reddit:plasma_glassy}}}
\frame{\pic[1]{plasma_moe}{رابط کاربری شخصی‌سازی شده \lr{Plasma (KDE)} با تم Moe روی توزیع KaOS~\cite{fig:reddit:plasma_moe}}}
\frame{\pic[1]{plasma_otto}{رابط کاربری شخصی‌سازی شده \lr{Plasma (KDE)} با تم Otto روی توزیع KaOS~\cite{fig:reddit:plasma_moe}}}
\frame{\pic[.7]{plasma_imagination}{رابط کاربری شخصی‌سازی شده \lr{Plasma (KDE)} با تم Imagination روی توزیع دبین~\cite{fig:reddit:plasma_moe}}}
\begin{frame}{ابزارهای جایگزین}
  به علت متن-باز بودن نرم‌افزارهای لینوکسی معمولاً آنها از قابلیت‌ها، پورت‌ها و قالب‌های در مقایسه با رقبا پشتیبانی می‌کنند.
  \vfill
  ابزارهای آزاد پیشنهادی جایگزین نرم‌افزارهای خصوصی:
  \begin{itemize}
    \item \lr{IDE}: \lr{KDevelop ،NetBeans ،Eclipse ،MonoDevelop} و Emacs
    \item بسته آفیس: LibreOffice و OpenOffice
    \item ویرایش گرافیک وکتور و بیت‌مپ: GIMP و Krita
    \item موتور سه بعدی: Blender
    \item مدیاپلیر: VLC و MPV
    \item ویرایشگر: \lr{Vim ،Nano ،Gedit} و Geany
    \item مرورگر: Firefox
    \item ویرایشگر فیلم: Kdenlive و Olive
    \item موتور حروف چینی و تألیف: {\TeX} (اسلاید حاضر با {\XeTeX} ساخته شده است.)
  \end{itemize}
\end{frame}
\note{
  مثلا گیمپ از پسوند ادوبی فتوشاپ پشتیبانی می‌کند.
  اوپن و لیبره آفیس از پسوندهای مایکروسافت آفیس پشتیبانی می‌کنند.
}
\section{محیط متنی}
\divider{محیط متنی}

\refs
\end{document}
