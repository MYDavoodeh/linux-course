\documentclass[professionalfonts]{beamer}
% Add references to notes
\makeatletter\defbeameroption{show only notes}[]{\beamer@notestrue\beamer@notesnormalsfalse}

% -----
% BEGIN Theme
% -----
% Originally from https://kbroman.org/blog/2013/10/07/better-looking-latexbeamer-slides/
\usetheme{default}
\usefonttheme{serif} % (Required for Persian)

\hypersetup{pdfpagemode=UseNone} % hides bookmarks on initial view
\beamertemplatenavigationsymbolsempty % removes navigation buttons clashing with the defined slide numbers

\setbeamertemplate{sections/subsections in toc}[sections numbered] % replaces bullets in toc with numbers
\setbeamertemplate{bibliography item}{\insertbiblabel} % Numbered bibliography
\setbeamertemplate{itemize subitem}{{\textendash}} % changes bullets to \textendash
% Make bullets/nums smaller
\setbeamerfont{itemize/enumerate subbody}{size=\footnotesize}
\setbeamerfont{itemize/enumerate subitem}{size=\footnotesize}
% Slide number
 \setbeamertemplate{footline}{%
   \raisebox{5pt}{\makebox[\paperwidth]{\hfill\makebox[20pt]{\color{gray}
         \scriptsize\insertframenumber}}}\hspace*{5pt}}

\definecolor{foreground}{RGB}{255,255,255}
\definecolor{background}{RGB}{24,24,24}
\definecolor{title}{RGB}{107,174,214}
\definecolor{gray}{RGB}{155,155,155}
\definecolor{subtitle}{RGB}{102,255,204}
\definecolor{hilight}{RGB}{102,255,204}
\definecolor{vhilight}{RGB}{255,111,207}

\setbeamercolor{titlelike}{fg=title}
\setbeamercolor{subtitle}{fg=subtitle}
\setbeamercolor{institute}{fg=gray}
\setbeamercolor{normal text}{fg=foreground,bg=background}
\setbeamercolor{item}{fg=foreground} % color of bullets
\setbeamercolor{subitem}{fg=gray}
\setbeamercolor{itemize/enumerate subbody}{fg=gray}
\setbeamercolor{section in toc}{fg=foreground}
\setbeamercolor{subsection in toc}{fg=gray}

\raggedleft % right aligns for Persian texts
% -----
% END Theme
% -----

\usepackage{amsmath} % for more symbols and mafs
\usepackage{hyperref} % for links

\usepackage{caption}
\captionsetup{justification=raggedleft} % right aligns multiline captions

\graphicspath{{../images/}}

\usepackage[backend=biber, sorting=none]{biblatex}
\addbibresource{../bib.bib}

\usepackage{xepersian} % must be the last package
\settextfont{XB Roya}
\setlatintextfont{Vazir}
\setdigitfont{XB Roya}
\setmonofont{Iosevka}

\author{محمدیاسین داوده}
\title{لینوکس}
\subtitle{هرآنچه لازم است بدانید}
\date{\today}

% -----
% BEGIN Helpers
% -----
\newcommand{\start}[1][نقشه راه]{
  \frame{\maketitle}
  \begin{frame}{#1}\tableofcontents\end{frame}
}

\newcommand{\bib}{
  \begin{LTR}
    \printbibliography[heading=none]
  \end{LTR}
}

\newcommand{\refs}{
  \section{مراجع}
  \begin{frame}{مراجع}
    \bib
  \end{frame}
  \note{\bib}
}

\newcommand{\subt}[1]{{\footnotesize\color{subtitle}{#1}}}
% -----
% END Helpers
% -----

\begin{document}
\start

% فارسی
\begin{frame}{سیستم عامل}
  \begin{figure}
    \includegraphics[width=.4\textheight]{os}
    \caption*{سیستم عامل رابط بین کاربر و سخت‌افزار~\cite{fig:wp:os}}
  \end{figure}
\end{frame}
\note{سیستم‌عامل خوب مثل دوربین خوب در یک بازیست. بهترین سیستم‌عامل آنی است که متوجه وجود آن نشوید. وظیفهٔ آن ارائه رابط و منابع به برنامه‌ها است. از دو بخش کرنل (رابط بین ابزارها و هسته سیستم) و ابزارهای مرتبط و سیستمی ساخته می‌شود.}
\section{تاریخچه}
\begin{frame}{یونیکس}
  \begin{figure}
    \includegraphics[width=.8\textwidth]{unix_plate}
    \caption*{سیستم عامل نوین یونیکس بر پایه C~\cite{fig:wp:unix_plate}}
  \end{figure}
\end{frame}
\note{
  ساختهٔ ۱۹۷۰ آزمایشگاه بل توسط \lr{Ken Thompson}, \lr{Dennis Ritche} و چندی دیگر.

  سیستم‌عامل خصوصی غالب دهه ۷۰. هزینه‌بر و کندتر از سیستم‌عامل‌های دورهٔ خودش.

  در آن دوره سیستم‌عامل‌ها با نرم‌افزارهای خاص خودشان و بعضاً با سخت‌افزارهای خودشان منتشر می‌شدند.
}
\begin{frame}{ریچارد متیو استالمن (R.M.S)}
  \begin{figure}
    \includegraphics[width=.8\textheight]{rms}
  \caption*{
    عکس ریچارد متیو استالمن بنیانگذار \textit{بنیاد نرم‌افزارهای آزاد (FSF)
      % \LTRfootnote{Free Software Foundation}
    } و \textit{پروژهٔ گنو} بر جلد \textit{آزاد مانند آزادی: جهاد ریچارد استالمن برای نرم‌افزار آزاد
      % \LTRfootnote{Free as in Freedom: Richard Stallman's Crusade for Free Software}
    } نوشتهٔ \textit{سم ویلیامز
      % \LTRfootnote{Sam Williams}
    }~\cite{fig:wp:rms}
  }
  \end{figure}
\end{frame}
\note{
  تا زمانی که مایکروسافت مفهوم نسبتاً جدید «نرم‌افزار خصوصی (\lr{Proprietory})» را در اواخر دهه ۷۰ مطرح کرد. نرم‌افزار تقریباً آزادانه قابل انتقال و اشتراک‌گذاری بود.

  استالمن سال ۷۱ وارد آزمایشگاه هوش مصنوعی MIT شد که به علت محدودیت‌های یونیکس سیستم‌عامل خود را ساخته بودند. نام آن سیستم اشتراک زمانی ناسازگار (Incompatible Timesharing System) بود. (ریچارد استالمن)

  اریک ریموند (مؤلف «کلیسا و بازار») نقل می‌کند که استالمن تجارب بدی با نرم‌افزارهای خصوصی آزمایشگاه MIT داشت (می‌خواست نرم‌افزارهایی که خریده بود را ویرایش کند اما شرکت اجازه حل مشکل را به او نمی‌داد).
  استالمن بر این باور بود که نگه داشتن کد و اشتراک نگذاشتن آن به مثل تک-خوری در دانش و توسعه است.

  استالمن بر آن شد تا بنیادی در خلاف این حقوق دست‌وپاگیر بنا کند و سیستم‌عاملی آزاد از این حقوق بنویسد. بنابراین بنیاد نرم‌افزارهای آزاد بنا شد.

  در ژانویه ۸۴ استفا داد و شروع به توسعه پروژه گنو کرد: GNU is Not Unix

 ابتدا به تنهایی و بعداً با کمک جامعه، یکی یکی تمام ابزارهای یونیکس را بازنویسی کرده و در انتها در صدد بازنویسی کرنل بودند.
  \cite{ros}
}

\refs
\end{document}
